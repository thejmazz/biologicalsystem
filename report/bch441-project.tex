\input{"preamble"}

\usepackage{tabularx}

\hypersetup{
  colorlinks=false,
  linkcolor=black,
  citecolor=black,
  urlcolor=cyan
}

\usepackage[round]{natbib}
\bibliographystyle{plainnat}

\graphicspath{{"./fig/"}}


\def \goml {\href{http://www.ebi.ac.uk/QuickGO/GTerm?id=GO:0051646}{GO:0051646 (mitochondrion localization)}}
\def \gomml {\href{http://www.ebi.ac.uk/QuickGO/GTerm?id=GO:0051659}{GO:0051659} (maintenance of mitochondrion localization)}
\def \gomert {\href{http://www.ebi.ac.uk/QuickGO/GTerm?id=GO:1990456}{GO:1990456} (mitochondrion-ER tethering)}
\def \goeml {\href{http://www.ebi.ac.uk/QuickGO/GTerm?id=GO:0051654}{GO:0051654} (establishment of mitochondrion localization)}
\def \gommaaf {\href{http://www.ebi.ac.uk/QuickGO/GTerm?id=GO:0034642}{GO:0034642} (mitochondrial migration along actin filament)}
\def \goemlmm {\href{http://www.ebi.ac.uk/QuickGO/GTerm?id=GO:0034643}{GO:0034643} (establishment of mitochondrial localization, microtubule mediated)}
\def \goemlma {\href{http://www.ebi.ac.uk/QuickGO/GTerm?id=GO:0034640}{GO:0034640} (establishment of mitochondrion localization by microtubule attachment)}
\def \gomtam {\href{http://www.ebi.ac.uk/QuickGO/GTerm?id=GO:0047497}{GO:0047497} (mitochondrion transport along microtubule) }
\def \goemlimf {\href{http://www.ebi.ac.uk/QuickGO/GTerm?id=GO:0090146}{GO:0090146} (establishment of mitochondrial localization involved in mitochondrial fission)}
\def \goremlimf {\href{http://www.ebi.ac.uk/QuickGO/GTerm?id=GO:0090147}{GO:0090147} (regulation of establishment of mitochondrion localization involved in mitochondrial fission)}
\def \gomd {\href{http://www.ebi.ac.uk/QuickGO/GTerm?id=GO:0048311}{GO:0048311} (mitochondrion distribution)}
\def \goidm {\href{http://www.ebi.ac.uk/QuickGO/GTerm?id=GO:0048312}{GO:0048312} (intracellular distribution of mitochondria)}
\def \gomi {\href{http://www.ebi.ac.uk/QuickGO/GTerm?id=GO:0000001}{GO:0000001} (mitochondrion inheritance)}

\begin{document}

\title{
\vspace{-155pt}
\hspace*{-80pt}\includegraphics[width=1.35\linewidth]{"purkinje-neuron-mitochondria"}
Neuronal Mitochondrion Trafficking \\
\small{\textit{BCH441 Project: Defining a System}}
}
\author{Julian Mazzitelli}
\date{Dec. 25, 2015}

\maketitle

\begin{center}
\textit{
The source code, notebook, and data pipeline can be found at
\href{https://github.com/thejmazz/biologicalsystem}{github.com/thejmazz/biologicalsystem}. \\
Cover image (mitochondrion in Purkinje neuron) by Atlas of Ultrastructural Neurocytology\footnote{\href{http://synapses.clm.utexas.edu/atlas/1_1_2_8.stm}{synapses.clm.utexas.edu/atlas/1\_1\_2\_8.stm}}
}
\end{center}

\begin{bottompar}
\section*{Introduction}

The ``powerhouse of the cell'' as it is so commonly called, the mitochondria is
one of the most vital organelles in eukaryotes. This structure is thought to
have developed through a symbiotic relationship among engulfed prokaryotic cells
and their hosts. As such, it is rooted quite deeply evolutionarily, and one
might expect its proper functioning to be absolutely vital, that is, knock-out
mutants will not survive. This is true - but as we will see, it is not just the
performance of this organelle which is centrally important, but where it is
localized within the cell as well.

\end{bottompar}

Images of isolated mitochondria were first observed in \citeyear{Lincoln1979}
by \cite{Lincoln1979}:

\begin{center}
  \includegraphics[width=0.6\linewidth]{rhodamine-mitochondria}
\end{center}

\noindent The variety of mitochondrion shape and size is clear, ranging from
globular to filamentous to networked structures. As well, the authors observed
movement during 15-30 sec intervals, between fluorescent and phase-contrast
photographs.

It goes without a citation to say that the primary role of a mitochondrion is to
supply energy to the cell in the form of ATP units, through the electron
transport chain among the cristae. Where is that energy needed? Consider highly
polar and elongated cells such as neurons. The cell body of a neuron is distant
from its synaptic endings, where as it happens, large amounts of energy are
required for neurotransmitter release and absorption. Following, we will
investigate the \textbf{system} whose \textbf{functional role} is the
\textbf{localization of mitochondrion within neurons}.

\section*{The System}

\begin{tabularx}{\linewidth}{l X}
  \textit{Name} & Localization/Trafficking of mitochondrion within neurons \\
  \textit{Description} & The collective of functional units represented by genes which process signals, transduce these events, initiate, and maintain the actions necessary to transport mitochondrion to distal points along the axon of a neuron. \\
  \textit{Associated GO Terms} & \goml
\end{tabularx}

\begin{itemize}
  \item \gomml
  \begin{itemize}
    \item \gomert
  \end{itemize}
  \item \goemlmm
  \begin{itemize}
    \item \gommaaf
    \item \goemlmm
    \begin{itemize}
      \item \goemlma
      \item \gomtam
    \end{itemize}
    \item \goemlimf
    \begin{itemize}
      \item \goremlimf
    \end{itemize}
  \end{itemize}
  \item \gomd
  \begin{itemize}
    \item \goidm
    \item \gomi
  \end{itemize}
\end{itemize}

Why this system? Originally I was looking into ``mitochondrial localization.''
Amongst the genes returned by the ontology, there appeared those related to
mitochondrial localization during cellular reproduction, transport,
microtubules, tethers, mRNA-binding, and various ``popular'' genes such as
ubiqutins, serum  albumin, leucine-rich repeat serine/threonine-protein kinase,
basic helix-loop-helix protein. There was a fair amount of variety. In order to
gather together a structured list of genes I would need to filter these out, and
to filter these out I would need a functional goal. I decided to choose the
neuronal process because it is one of the most extreme cases of mitochondrial
movement in all cell types, there was a decent amount of related literature
available, some elements of its processes had been recently elecidated, and it
has important neurophysiological consequences. A review by \cite{Reis2009}
explored the atypical Miro GTPases and their role in transporting mitochondria
in neurons. The authors note that aberrant mitochondrial dynamics can contribute
to Amyotropic Lateral Sclerosis (ALS), Huntington's, Parkinson's, and
Alzheimer's diseases. A more recent experiment by \cite{Loss2015} examines the
role of TRAK1 and TRAK2 kinesin adaptor proteins which link mitochondria to
kinesin motor proteins. Furthermore, Miro proeins are expressed in a large
variety of cell types, extending this current analysis to new domains
\citep{Reis2009}.

\subsection*{System Role Ontology}

With this goal in mind, I fetched, filtered, and discovered
a set of genes for which are involved and cooperate in achieving this functional
role.

\section*{Gene Collection}

\bibliography{biblio}


\end{document}
