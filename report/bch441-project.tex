\input{"preamble"}

\usepackage{tabularx}

\hypersetup{
  colorlinks=true,
  linkcolor=black,
  citecolor=black,
  urlcolor=cyan
}

\usepackage[round]{natbib}
\bibliographystyle{plainnat}

\graphicspath{{"./fig/"}}


\def \goml {\href{http://www.ebi.ac.uk/QuickGO/GTerm?id=GO:0051646}{GO:0051646} (mitochondrion localization)}
\def \gomml {\href{http://www.ebi.ac.uk/QuickGO/GTerm?id=GO:0051659}{GO:0051659} (maintenance of mitochondrion localization)}
\def \gomert {\href{http://www.ebi.ac.uk/QuickGO/GTerm?id=GO:1990456}{GO:1990456} (mitochondrion-ER tethering)}
\def \goeml {\href{http://www.ebi.ac.uk/QuickGO/GTerm?id=GO:0051654}{GO:0051654} (establishment of mitochondrion localization)}
\def \gommaaf {\href{http://www.ebi.ac.uk/QuickGO/GTerm?id=GO:0034642}{GO:0034642} (mitochondrial migration along actin filament)}
\def \goemlmm {\href{http://www.ebi.ac.uk/QuickGO/GTerm?id=GO:0034643}{GO:0034643} (establishment of mitochondrial localization, microtubule mediated)}
\def \goemlma {\href{http://www.ebi.ac.uk/QuickGO/GTerm?id=GO:0034640}{GO:0034640} (establishment of mitochondrion localization by microtubule attachment)}
\def \gomtam {\href{http://www.ebi.ac.uk/QuickGO/GTerm?id=GO:0047497}{GO:0047497} (mitochondrion transport along microtubule) }
\def \goemlimf {\href{http://www.ebi.ac.uk/QuickGO/GTerm?id=GO:0090146}{GO:0090146} (establishment of mitochondrial localization involved in mitochondrial fission)}
\def \goremlimf {\href{http://www.ebi.ac.uk/QuickGO/GTerm?id=GO:0090147}{GO:0090147} (regulation of establishment of mitochondrion localization involved in mitochondrial fission)}
\def \gomd {\href{http://www.ebi.ac.uk/QuickGO/GTerm?id=GO:0048311}{GO:0048311} (mitochondrion distribution)}
\def \goidm {\href{http://www.ebi.ac.uk/QuickGO/GTerm?id=GO:0048312}{GO:0048312} (intracellular distribution of mitochondria)}
\def \gomi {\href{http://www.ebi.ac.uk/QuickGO/GTerm?id=GO:0000001}{GO:0000001} (mitochondrion inheritance)}

\def \SyRO {\href{https://github.com/hyginn/SyRO}{github.com/hyginn/SyRO}}
\def \firstpool {\href{https://github.com/thejmazz/biologicalsystem/blob/master/notebook.md\#user-content-summary-of-first-pool}{notebook}}

\newcommand{\uniprot}[1]{\href{http://www.uniprot.org/uniprot/#1}{#1}}

\begin{document}

\title{
\vspace{-155pt}
\hspace*{-80pt}\includegraphics[width=1.35\linewidth]{"purkinje-neuron-mitochondria"}
Neuronal Mitochondrion Trafficking \\
\small{\textit{BCH441 Project: Defining a System}}
}
\author{Julian Mazzitelli}
\date{\today}

\maketitle

\begin{bottompar}
\begin{center}
\textit{
The source code, notebook, and data pipeline can be found at
\href{https://github.com/thejmazz/biologicalsystem}{github.com/thejmazz/biologicalsystem}. \\
Cover image (mitochondrion in Purkinje neuron) by Atlas of Ultrastructural Neurocytology\footnote{\href{http://synapses.clm.utexas.edu/atlas/1_1_2_8.stm}{synapses.clm.utexas.edu/atlas/1\_1\_2\_8.stm}}
}
\end{center}
\end{bottompar}

\tableofcontents


\section{Introduction}

The ``powerhouse of the cell'' as it is so commonly called, the mitochondria is
one of the most vital organelles in eukaryotes. This structure is thought to
have developed through a symbiotic relationship among engulfed prokaryotic cells
and their hosts. As such, it is rooted quite deeply evolutionarily, and one
might expect its proper functioning to be absolutely vital, that is, knock-out
mutants will not survive. This is true - but as we will see, it is not just the
performance of this organelle which is centrally important, but where it is
localized within the cell as well.

Images of isolated mitochondria were first observed in \citeyear{Lincoln1979}
by \citeauthor{Lincoln1979}:

\begin{center}
  \includegraphics[width=0.6\linewidth]{rhodamine-mitochondria}
\end{center}

\noindent The variety of mitochondrion shape and size is clear, ranging from
globular to filamentous to networked structures. As well, the authors observed
movement during 15-30 sec intervals, between fluorescent and phase-contrast
photographs.

The primary role of a mitochondrion is to supply energy to the cell in the form
of ATP units, through the electron transport chain among the cristae. Where is
that energy needed? Consider highly polar and elongated cells such as neurons.
The cell body of a neuron is distant from its synaptic endings, where as it
happens, large amounts of energy are required for neurotransmitter release and
absorption. Following, we will investigate the \textbf{system} whose
\textbf{functional role} is the \textbf{localization of mitochondrion within
neurons}.

\section{The System}

\begin{tabularx}{\linewidth}{l X}
  \textit{Name} & Localization/Trafficking of mitochondrion within neurons \\
  \textit{Description} & The collective of functional units represented by genes which process signals, transduce these events, initiate, and maintain the actions necessary to transport mitochondrion to distal points along the axon of a neuron. \\
  \textit{Associated GO Terms} & \goml
\end{tabularx}

\begin{itemize}
  \item \gomml
  \begin{itemize}
    \item \gomert
  \end{itemize}
  \item \goemlmm
  \begin{itemize}
    \item \gommaaf
    \item \goemlmm
    \begin{itemize}
      \item \goemlma
      \item \gomtam
    \end{itemize}
    \item \goemlimf
    \begin{itemize}
      \item \goremlimf
    \end{itemize}
  \end{itemize}
  \item \gomd
  \begin{itemize}
    \item \goidm
    \item \gomi
  \end{itemize}
\end{itemize}

Why this system? Originally I was looking into ``mitochondrial localization.''
Amongst the genes returned by the ontology, there appeared those related to
mitochondrial localization during cellular reproduction, transport,
microtubules, tethers, mRNA-binding, and various ``popular'' genes such as
ubiqutins, serum  albumin, leucine-rich repeat serine/threonine-protein kinase,
basic helix-loop-helix protein. There was a fair amount of variety. In order to
gather together a structured list of genes I would need to filter these out, and
to filter these out I would need a functional goal. I decided to choose the
neuronal process because it is one of the most extreme cases of mitochondrial
movement in all cell types, there was a decent amount of related literature
available, some elements of its processes had been recently elecidated, and it
has important neurophysiological consequences. A review by \cite{Reis2009}
explored the atypical Miro GTPases and their role in transporting mitochondria
in neurons. The authors note that abnormal mitochondrial dynamics can contribute
to Amyotropic Lateral Sclerosis (ALS), Huntington's, Parkinson's, and
Alzheimer's diseases. A more recent experiment by \cite{Loss2015} examines the
role of TRAK1 and TRAK2 kinesin adaptor proteins which link mitochondria to
kinesin motor proteins. Furthermore, Miro proeins are expressed in a large
variety of cell types, potentially extending this current analysis to new
domains \citep{Reis2009}.

\subsection{Systems Role Ontology}

To define this system in a structured manner, I considered its functionality
in the context of the Systems Roles Ontology, which can be found at \SyRO:

\begin{center}
  \includegraphics[width=1\linewidth]{SyRO-2015-10-27}
\end{center}

\begin{tabularx}{\linewidth}{l l X}
  \textit{\textcolor{grey}{Name}} & \textit{\textcolor{grey}{ID}} & \textit{\textcolor{grey}{Context Within Neuronal Mitochondrion Localization}} \\
  \textbf{input} & \texttt{16} & compounds or signals prompting the directed movement of mitochondrion \\
  \textbf{integrate signals} & \texttt{22} & machinery which directs input signals/compounds to the system \\
  \textbf{output} & \texttt{21} & mitocondrion transport towards synaptic endings \\
  \textbf{regulate} & \texttt{20} & components which ensure regular mitochondrial motility \\
  \textbf{set} & \texttt{19} & preparing the functional units, ``setting the stage'' as it were  \\
  \textbf{transform} & \texttt{18} & altering the system status so that it may be reversed, stopped, reinstated \\
  \textbf{transmit} & \texttt{17} & the physical machinery to move mitochondria \\
\end{tabularx} \\

These functions collaborate to produce the \textbf{functional role} of
localizing mitochondria \textbf{for the proper functioning of neural
communication}. I will define the bounds of this system as any genes which can
be annotated with the functional role annotations above. Structural role
annotations will be considered second to function, and will be used largely to
describe how that component physically determines its function. With this goal
in mind, I fetched and filtered data from the Gene Ontology Consortium, QuickGO,
UNIPROT, IntAct, and STRING. Those genes which did not make the cut can be seen
through the \textit{Summary of first pool} in my \firstpool.

\section{Gene Collection}

\rowcolors{2}{white}{gray!25}
\begin{tabularx}{\linewidth}{l l l X}
  \textit{Gene} & \textit{Accession} & \textit{Name} & \textit{SyRO} \\
  ATCAY & \uniprot{Q86WG3} & Caytaxin & regulate \\
  BHLHA15 & \uniprot{Q7RTS1} & Class A basic helix-loop-helix protein 15 & transmit \\
  CLUH & \uniprot{I3L2B0} & Clustered mitochondria protein homolog & set \\
  GABA & \uniprot{P80404} & Gamma-amino-N-butyrate transaminase & transmit \\
  GAN & \uniprot{Q9H2C0} & Gigaxonin & transform \\
  KIF1B & \uniprot{O60333} & Kinesin-like protein KIF1B & output \\
  KIF5B & \uniprot{P33176} & Kinesin-1 heavy chain & output \\
  KLCA1 & \uniprot{Q07866} & Kinesin light chain 1 & output \\
  MAPT & \uniprot{P10636} & Microtubule-associated protein tau & set \\
  MAP1B & \uniprot{P46821} & Microtubule-associated protein 1B & set \\
  MGARP & \uniprot{Q8TDB4} & Mitochondria-localized glutamic acid-rich protein & regulate \\
  MTM1 & \uniprot{Q13496} & Myotubularin & set \\
  MSTO1 & \uniprot{Q9BUK6} & Protein misato homolog 1 & regulate \\
  RHOT1 & \uniprot{Q8IXI2} & Mitochondrial Rho GTPase 1 & integrate signals \\
  RHOT2 & \uniprot{Q8IXI1} & Mitochondrial Rho GTPase 2 & integrate signals \\
  SNPH & \uniprot{O15079} & Syntaphilin & regulate \\
  SYBU & \uniprot{Q9NX95} & Syntabulin & set \\
  TIAM2 & \uniprot{Q8IVF5} & T-lymphoma invasion and metastasis-inducing protein 2 & integrate signals \\
  TRAK1 & \uniprot{Q9UPV9} & Trafficking kinesin-binding protein 1 & transmit \\
  TRAK2 & \uniprot{Q8IU62} & Trafficking kinesin-binding protein 2 & transmit \\
  TTL & \uniprot{Q8NG68} & Tubulin--tyrosine ligase & transform \\
\end{tabularx}

\subsection{Signal Integration}

\subsubsection{RHOT1, RHOT2 \textit{Mitochondrial Rho GTPase 1, 2}}

RHOT1 and RHOT2, which are also known as MIRO1 and MIRO2, were first reported as
a new family of Rho GTPases with in \citeyear{Fransson2003} by
\citeauthor{Fransson2003}. An NCBI CDD search presents us with a two GTPase
domains each terminus, with two EF hands in between. The C terminal TM domain
targets this protein to the mitochondria.

\begin{figure}[h]
  \includegraphics[width=1.0\linewidth]{Rhot-CDD}
\end{figure}

NTPases are a large superfamily and operate in a large variety of systems. The
on/off state of these proteins confers their ability to integrate signals into a
pathway. In this way we can observe that whichever is activiting Rhot, Rhot is
integrating that signal downwards to its effector. \citeauthor{Fransson2003}
also observed an interesting property: overexpression of Miro1/Val-13 led to
an aggregation of the mitochondrial network. The same authors separated this
response into two distinct phenotypes later in \citeyear{Fransson2006}. They
observed that Miro-1 induced aggregration and thread-like mitochondria, whereas
Miro-2 only induced aggregation. \citeauthor{Fransson2006} also demonstrated
interactions of Miro with GRIF-1 (TRAK2) and OIP106 (TRAK1), trafficking
kinesin binding proteins.

% Ability to bind to TRAK1 not Ca2+ dep.

% TODO move to regulation?
\subsubsection{TIAM2 \textit{T-lymphoma invasion and metastasis-inducing protein 2}}

TIAM2 regulates the activity of RHO-like proteins (UNIPROT). This is confirmed
with the existence of a GEF domain from the CDD.

\begin{figure}[h]
  \includegraphics[width=1.0\linewidth]{TIAM2-CDD}
\end{figure}

In this way TIAM2 has the ability to activate Rhot. TIAM2 has also been shown to
promote the migration of neurons in the cerebral cortex (UNIPROT).

\subsection{Set}

\subsubsection{CLUH \textit{Clustered mitochondria protein homolog}}

A CDD search with CLU1 from yeast presents the CLU domain (CLUstered
mitochondria). This domain is required for mitochondrial positioning and
transport; improper function can lead to mitochondrion clustering at the
microtubule plus ends (CDD).

\subsubsection{MAP1B \textit{Microtubule-associated protein 1B}}

By similarity, MAP1B facilitates tyrosination of $\alpha$-tubulin in
neuronal microtubules. Interacts with TIAM2 and TTL (UNIPROT). IntAct
suggests an interaction with GAN.

\begin{center}
  \includegraphics[width=0.5\linewidth]{MAP1B-IntAct}
\end{center}

\subsubsection{MAPT \textit{Microtubule-associated protein tau}}

CDD for MAPT presents tubulin binding repeat domains. MAPT is expected to
stablize microtubules and potentially establish and maintain neuronal polarity
(UNIPROT).

\begin{center}
  \includegraphics[width=1.0\linewidth]{MAPT-CDD}
\end{center}

\subsubsection{MTM1 \textit{Myotubularin}}

\cite{Hnia2011} observed that decreased MTM1 expression and mutations
induced abnormal mitochondrial positioning, shape, dynamics, and function.

\subsubsection{SYBU \textit{Syntabulin}}

SYBU belongs to a kinesin motor-adapter complex. It is critical for forward
axonal transport (UNIPROT). STRING demonstrates binding interactions with
KIF5B, which in turn binds with KIF5A and TRAK2, of which both bind TRAK1.

\begin{center}
  \includegraphics[width=0.5\linewidth]{SYBU-STRING}
\end{center}

\subsection{Transmit}

\subsubsection{BHLHA15 \textit{Class A basic helix-loop-helix protein 15}}

Required for mitochondrial calcium ion transport (UNIPROT).

\subsubsection{GABA \textit{Gamma-amino-N-butyrate transaminase}}

GABA is located within the mitochondrial matrix (UNIPROT).

\subsubsection{TRAK1 \textit{Trafficking kinesin-binding protein 1}}

Trafficking of GABA-A receptors (UNIPROT). Contains coiled-coiled domain. STRING
corroborates with interactions for SYBU above.

\begin{center}
  \includegraphics[width=0.5\linewidth]{SYBU-STRING}
\end{center}

\subsubsection{TRAK2 \textit{Trafficking kinesin-binding protein 2}}

% TODO explore Milton
Milton domain. Interact with GABA-A receptors.

\subsection{Transform}

\subsubsection{GAN \textit{Gigaxonin}}

Controls degredation of MAP1B and MAP1S; critical for neuronal maintenance and
surival (UNIPROT).

\subsubsection{TTL \textit{Tubulin--tyrosine ligase}}

ATP + detyrosinated alpha-tubulin + L-tyrosine = alpha-tubulin + ADP + phosphate

\subsection{Regulate}

\subsubsection{ATCAY \textit{Caytaxin}}

May regulate the localization of mitochondria within axons and dendrites
(UNIPROT).

\subsubsection{MGARP \textit{Mitochondria-localized glutamic acid-rich protein}}

Regulates kinesin-mediated axonal transport of mitochondria to nerve terminals.
Translocation of TRAK2 from cytoplasm to mitochondrion.

\subsubsection{MSTO1 \textit{Protein misato homolog 1}}

Regulation of mitochondrial dist. and morphology.

\subsubsection{SNPH \textit{Syntaphilin}}

Inhibits SNARE complex formation by absorbing free syntaxin-1.

\subsection{Output}

\subsubsection{KIF1B \textit{Kinesin-like protein KIF1B}}

MT + end directed motility.

\subsubsection{KIF5B \textit{Kinesin-1 heavy chain}}

Kinesin heavy chain.

\subsubsection{KLC1 \textit{Kinesin light chain 1}}

Kinesin light chain.

\bibliography{biblio}


\end{document}
